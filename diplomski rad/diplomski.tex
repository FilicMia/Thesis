% Predlozak za pisanje diplomskog rada na PMF-MO
% Opcenita uputstva za LaTeX se mogu npr. naci na 
% http://web.math.hr/nastava/rp3, http://web.math.hr/nastava/s4-prof/latex.pdf
% NE PREPORUCA se "Ne baš tako kratak uvod u TEX", buduci se radi o vrlo starom prirucniku
% koji nije pogodan za moderne verzije LaTEXa.
% Originalna verzija "The not so short..." na http://tobi.oetiker.ch/lshort/lshort.pdf 
% je obnovljena i daje bolji uvid u moderne verzije LaTeXa

% Stil je optimiziran za kreiranje pdf dokumenta (npr. pomocu pdflatex-a, XeLaTeX-a)

\documentclass[a4paper,twoside,12pt]{memoir} % jednostrano: promijeniti twoside u oneside

% Paket inputenc omogucava direktno unosenje hrvatskih dijakritickih znakova 
% opcija utf8 za unicode (unix, linux, mac)
% opcija cp1250 za windowse
\usepackage[utf8]{inputenc}  % ukoliko se koristi XeLaTeX onda je \usepackage{xunicode}\usepackage{xltxtra}

% Stil za diplomski, unutra je ukljucena podrska za hrvatski jezik
\usepackage{diplomski}
% bibliografija na hrvatskom
\usepackage[languagenames,fixlanguage,croatian]{babelbib} % zahtijeva datoteku croatian.bdf
% hiperlinkovi 
\usepackage[pdftex]{hyperref} % ukoliko se koristi XeLaTeX onda je \usepackage[xetex]{hyperref}

% Odabir familije fontova:
% koristenjem XeLaTeX-a mogu se koristiti svi fontovi instalirani na racunalu, npr
% \defaultfontfeatures{Mapping=tex-text}
% \setmainfont[Ligatures={Common}]{Hoefler Text}
% ili
% \newcommand{\nas}[1]{\fontspec{Adobe Garamond Pro}\fontsize{24pt}{24pt}\color{Chocolate}\selectfont #1}
% i onda \nas{Naslov ...}
\usepackage[pdftex]{graphicx}
\graphicspath{ {images/} }
\usepackage{float}
\usepackage{wrapfig}
\usepackage{url}
\usepackage{lipsum}
\usepackage{hyperref}
\usepackage{cleveref}
\usepackage{epstopdf}
\usepackage{multirow}
\usepackage{caption}
\usepackage[table]{xcolor}
\PassOptionsToPackage{hyphens}{url}\usepackage{hyperref}
\usepackage[linesnumbered,ruled,noline]{algorithm2e}

\renewcommand*{\listalgorithmcfname}{Lista algoritama}
\renewcommand*{\algorithmcfname}{Algoritam}
\renewcommand*{\algorithmautorefname}{algoritam}

% Paket graphicx sluzi za manipuliranje grafikom 
 % ukoliko se koristi XeLaTeX onda je \usepackage[xetex]{graphicx}
% Paket amsmath je vec ukljucen
% Dodatno definirane matematicke okoline:
% teorem (okolina: thm), lema (okolina: lem), korolar (okolina: cor),
% propozicija (okolina: prop), definicija (okolina: defn), napomena (okolina: rem),
% slutnja (okolina: conj), primjer (okolina: exa), dokaz (okolina: proof)
% Definirane su naredbe za ispisivanje skupova N, Z, Q, R i C
% Definirane su naredbe za funkcije koje se u hrvatskoj notaciji oznacavaju drukcije 
% nego u americkoj: tg, ctg, ... (\tgh za tangens hiperbolni)
% Takodjer su definirane naredbe za Ker i Im (da bi se razlikovala od naredbe za imaginarni dio kompleksnog
% broja, naredba se zove \slika).

\pagestyle{headings}
% uz paket fancyhdr mogu se lako kreirati fancy zaglavlja i podnozja

% Podaci koje treba unijeti
\title{Analiza postupka procjene položaja temeljem
	zadanih pseudoudaljenosti u programski određenom
	prijamniku za satelitsku navigaciju}
\author{Mia Filić}
\advisor{izv. prof. dr. sc Luka Grubišić i prof. dr. sc Renato Filjar}  % obavezno s titulom (prof. dr. sc ili doc. dr. sc.)
\date{\today}  % oblika mjesec, godina

% Moguce je unijeti i posvetu
% Ukoliko nema posvete, dovoljno je iskomentirati/izbrisati sljedeci redak 
\dedication{Na kraju}

\begin{document}

% Naredna frontmatter generira naslovnu stranicu, stranicu za potpise povjerenstva, eventualnu posvetu i sadrzaj
% Moze se iskomentirati ukoliko nije u pitanju konacna verzija
\frontmatter

% Tekst diplomskog ...

% Diplomski rad treba poceti s uvodnim poglavljem  
\begin{intro}
	Globalni navigacijski satelitski sustav (GNSS) je osmišljen s ciljem da
	u bilo kojem trenutku i za bilo koji entitet na Zemljinoj površini može dati podatak
	o trenutnom vremenu, položaju i brzini gibanja (engl. Position, Velocity and Time (PVT)). 
	Kao takav daje temelje rastućem broju tehnoloških i društveno-ekonomskih sustava.
	%DOVRŠITI
\end{intro}

%\chapter[Naslov poglavlja u sadržaju][Kratki naslov poglavlja]{Naslov poglavlja}	
% ukoliko naslov nije jako dugacak dovoljno je samo \chapter{Naslov poglavlja} 

%\section[Naslov sekcije u sadržaju][Kratki naslov sekcije]{Naslov sekcije}
%\subsection{Naslov podsekcije}

\chapter[Globalni navigacijski satelitski sustav (GNSS)][GNSS]{Globalni navigacijski satelitski sustav (engl. Global Navigation Satellite System (GNSS))}
	
	\begin{figure}[h]
		\centering
		\includegraphics[width=0.6\textwidth]{pictureNav}
		\caption{Satelitska navigacija\cite{bookProcessing} }
		\label{Fig:nn}
		
	\end{figure}
	%http://www.academia.edu/12873735/PRIMJENA_GPS_GLOBALNI_NAVIGACIJSKI_SISTEM_i_GNSS_GLOBALNI_NAVIGACIJSKI_SATELITSKI_SISTEM_U_GEOLO%C5%A0KOM_KARTIRANJU_I_IZRADI_IN%C5%BDENJERSKO-GEOLO%C5%A0KIH_KARATA_NA_PRIMJERU_KLIZI%C5%A0TA_JUNUZOVI%C4%86I_SREBRENIK
	%primjena u kartografiji
	Spominjući GNSS, najčešće se misli na \textit{"sazviježđe"}
	satelita koji odašilju signale, potrebne za određivanje trenutne cije, i \textit{Navigacijske poruke} (engl. Navigation Messages (NM)).
	\textit{"Sazviježđe"} satelita predstavlja (1) svemirski segment GNSS sustava.
	Postoji još (2) kontrolni segment koji čine kontrolne stanice smještene na Zemlji i (3) korisnički segment, tj. GNSS prijemnici \ref{Fig:GNSSsegmenti}.
	Kontrolni segment nadzire i upravljaja radom sustava.
	
	\begin{figure}[h]
			\centering
			\includegraphics[width=0.6\textwidth]{GNSSsegmenti}
			\caption{Segmenti GNS sustava (GNSS)} %https://dr.nsk.hr/islandora/object/fpz%3A511/datastream/PDF/view
			\label{Fig:GNSSsegmenti}
			
	\end{figure}
	
	Trenutno postoji više GNS sustava (GNSS). Neki su u potpunosti 
	operativni, dok su neki samo djelomično.
	Najraširaniji u civilnoj upotrebi je GPS (Global Positioning System).
	GPS je u potpunosti operativan i u vlasništnu Vlade SAD-a. Njime upravlja Ministarstvo obrane SAD-a (engl. US Air Force).
	GPS omogućava dvije znatno različite razine korištenja, civilnu i vojnu.
	Vojna razina korištenja ima više mogućnosti, a dopuštena je samo određenim 
	korisnicima. Civilna razina korištenja je dostupna svima, bez dodatne naknade, uz uvjet posjedovanja GPS-prijemnika. 
	
	Drugi, također u potpunost operativan GNSS, je GLONASS (Global'naya Navigatsionnaya Sputnikovaya Sistema) koji je u vlasništvu Rusije.
	Postoje i GNSS sustavi u razvoju. Jedan od njih je Galileo.
	Galileo-om upravlja Europska unija (EU). Galileo je najavljen postati u potpunosti operativan do 2020 \cite{bookProcessing}.
	Kina posjeduje lokalni navigacijski satelitski sustav BeiDou koji je najavljan postati svjetski do 2020 ~\cite{bookProcessing}.
	
	Primjena GNSS-a dijeli se na pozicioniranje i navigaciju.
	\begin{definition}[Navigacija]
		Navigacija obuhvaća trenutno određivanje položaja i brzine entiteta u pokretu.
		Svrha navigacije je praćenje i upravljanje gibanja entiteta.
	\end{definition}
	
	\begin{definition}[Pozicioniranje]
		Pozicioniranje nazivamo postupak određivanja položaja točkovnog entiteta ili niza 
		točkovnih entiteta u prostoru.
	\end{definition}
	Ovaj rad se bavi isključivo off-line navigacijskom primjenom, u svrhu praćenja entiteta.
	Off-line navigacijska primjena je važna za prometnu znanost u svrhu analize prometnih puteva. Kako ne zahtjeva real-time izračunavanje,
	svodi se na određivanje položaja točkovnog entiteta koji je statičan u danom vremenu $t$, tj. pozicioniranje.
	Određujući položaj entiteta za niz vremena $ t_1,t_2, \hdots ,t_n $, dobiva se 
	aproksimacija kretanja entiteta u vremenskom okviru $[t_1,t_n]$.
	Preciznost aproksimacije kretanja zadaje se veličinom okvira i parametrom $n$, ili dostupnošću podataka.
	Praksa ne zathjeva se da je $n$ u odnosu na vremenski okvir od 1 sata prevelik.
	Točno kretanje entiteta moguće je
	odrediti preslikavanjem dobivene aproksimacije na kartu prometnih puteva.
	U tu se svrhu koriste koriste otprije poznati algoritmi.
	Zaključno, bavimo se isključivo algoritmom za pozicioniranje (statičkog entiteta).
	
	U danjim poglavljima, baziramo se na jedan određeni GNSS, GPS u aspektu civilne razine korištenja.
	
\chapter[Globalni pozicijski sustav (GPS)][GPS]{Globalni pozicijski sustav (GPS, engl. Global Positioning System)}
	Sazvježđe GPS-a se sastoji od najmanje 24 satelita raspoređenih u 6 jednako odmaknutih orbita, svaka s inklanacijom od $55$ stupnjeva od ekvatorijalne
	ravnine (engl. Medium Earth Orbit (MEO)).
	Sateliti kruže na visini od oko 20200 kilometara od Zemljine površine s periodom rotacije 12 zvjezdanih sati. 
	Sateliti su raspoređeni na način tako da u svakom trenutku za svako mjesto na Zemljinoj površini postoje barem 4 dostupna satelita. Definicija dostupnosti satelita je dana na stranici \pageref{stranica:dostupnost}.
	
	Svi GPS sateliti odašilju signale na istoj osnovnoj frekvenciji/frekvencijama (Slika \ref{Fig:GPSSignal}). 
	U satelitima, vrijeme se prati pomoću cezijevih satova koji se sinkroniziraju s GPS atomskom vremenskom skalom. Sinkronizacija se odvija u periodima.
	\begin{figure}[H]
		\centering
		\includegraphics[width=0.4\textwidth]{GPS_Signals}
		\caption{GPS signal i njegove komponente \cite{GPS:1}}
		\label{Fig:GPSSignal}
	\end{figure}
	
	\section[C/A PRN kod]{GPS signali: C/A PRN i P kod}
	\subsection{C/A PRN kod i primjene}\label{CAkod}
	GPS sateliti odašilju signale na dvije frekvencije (nosača) $L_1$ i $L_2$, od kojih $L_1$ na 1575.42 MHz je namjenjena civilnoj upotrebi. Pojam signal se u satelitskoj navigaciji
	koristi za poruku koja sadržava C/A PRN kod (eng. Coarse Acquisition Pseudo Random Noise). 
	Svaki satelit posjeduje jedinstveni C/A PRN kod koji predstavlja niz od 1023 0 i 1.
	GPS-prijemnik razlikuje signale i \textit{Navigacijske poruke} različitih satelita
	temeljem sadržanih C/A PRN kodova. Satelit C/A PRN kodove odašilje neprestano, s početkom na početku svake sekunde. Prijemnik primljeni C/A PRN kod korsti za 
	razlikovanje satelita odašiljetelja, ali i za računanje pseudo-udaljenosti.
	
	\begin{definition}[Pseudo-udaljenost]
		Naka su svi sateliti numerirani brojevima iz $\mathbb{N}$ s početkom 
		u 1. Naka je $i \in \mathbb{N}$ neki satelit i $t$ prijamnik prijemnik
		koji je u mogućnosti primiti signal koji odašilje satelit $i$. Pseudo-udaljenost
		između satelita odašiljatelja $i$ i prijemnika primatelja $t$:
		$$d_i = c\cdot(t'_i- t_i)$$
		gdje je c konstanta koja je jednaka brzini putovanja signala od satelita do prijemnika. $t'_i$ je vrijeme primanja signala, $t_i$ vrijeme slanja signala
		(po UTC vremenu).
	\end{definition}
	\textbf{Pseudo-udaljenost} je aproksimacija udaljenosti između satelita odašiljatelja i prijemnika primatelja signala u danom trenutku.
	Označimo $\delta t := (t'_i- t_i)$. Izračun vremena putovanja signala se odvija poravnavanjem odgovarajućih signala, tj. C/A PRN kodova.
	U isto vrijeme, prijemnik i satelit generiraju isti C/A PRN kod. Budući da 
	dok signala putuje, prijemnik još uvijek generira C/A PRN kod, po primitku signala,
	ta 2 koda se uspoređuju. Temeljem promjene u fazi, dobinenog i generiranog signala,
	računa se procjena vremena putovanja, tj. $\delta t$ (Slika \ref{fig:deltat}).
	\begin{figure}[H]
		\centering
		\includegraphics[width=0.4\textwidth]{deltat}
		\caption{Procjena vremena putovanja signala ($\delta t$)}
		\label{fig:deltat}
	\end{figure}
	Za konstantu $c$ se obično uzima brzina svjetlosti koja predstavlje brzinu putovanja poruke satelita u vakuumu.
	
	Budući da se psudo-udaljenost dobiva poravnavanjem kodova \label{stranica:kodno},
	upravo opisani način određivanja pseudo-udaljenosti naziva se kodni.
	
	Postoji još i fazni način određivanja psudo-udaljenosti koji se zasniva na poravnavanju valova nosača (engl. Carrier phase) nakon micanjem C/A PRN i P(Y) kodova  iz poruke ( Slika \ref{Fig:GPSSignal} ).
	Fazno mjerenje služi kao nadopuna kodnome u svrhu poboljšanja točnosti određivanja položaja.
	
	
	\section{P kod}\label{Pkod}
	P kod se odašilje na obje frekvencije i rezerviran je za vojnu razinu upotrebe.
	Kao i C/A PRN kod, sastoji se od niza nula i jedinica i šalje se brzinom 1023 bit/s, ali je puno dulji.
	Potrebno je 37 tjedana kako bi se poslao cjelokupan P kod.
	Za razliku od C/A koda, gdje svaki satelit ima svoj jednistveni C/A kod, P kod je
	distribuiran među satelitima. Isječci P koda različitih satelita su međusobno različiti.
	U određeno vrijeme, svakih 7 dana, određeni satelit odašilje svoj dio P koda.
	Na taj način, prijemnik razlikuje jedan satelit od drugoga. Npr. ukoliko
	satelit $\mathnormal{S}$ odašilje 14. tjedan P koda, onda je satelit $\mathnormal{S}$
	zapravo \textit{Space Vehicle 14 (SV 14)}.
	Prijemnik ne prima goli P kod, već njegovu kriptiranu verziju, u oznaci $P(Y)$.
	Samo korisnici s vojnom razinom upotrebe su u mogućnosti dekriptirati $P(Y)$ u $P$.
	P kod omogućava točnije određivanje pozicije entiteta.
	
	\section{Pogreške određivanja položaja i vrste}\label{sec:pogreske}
	Pogreške određivanja položaja se grubo dijele na dvije vrste: (1)
	pogreške nastale usljed konstrukcije ulaza u algoritam i
	(2) usljed primjenje algoritma za određivanje položaja na mjerenim psudo-udaljeniostima.
	Dakle, postoje dva izvora: ulazni podatci algoritma (tip 1) i algoritam (tip 2).
	Izvori pogreške tipa 1 su najčešće pogreške pri određivanju pseudo-udaljenosti
	ili raspoređenost satelita u svemiru.
	Nepovoljan položaj promatranih satelita može rezultirati skoro pa zavisnom jednadžbama
	u \eqref{eq:1} (Slike \ref{fig:DOP}, \ref{fig:DOPLow} i \ref{fig:DOPHigh}).
	\begin{figure}[H]
		\centering
		\includegraphics[width=0.6\textwidth]{DOP}
		\caption{Razlike u razmještaju satelita}
		\label{fig:DOP}
	\end{figure}%
	\begin{figure}[H]
		\centering
		\includegraphics[width=0.6\textwidth]{DOPLow}
		\caption{Loš razmještaj satelita}
		\label{fig:DOPLow}
	\end{figure}%
	\begin{figure}[H]
		\centering
		\includegraphics[width=0.4\textwidth]{DOPHigh}
		\caption{Dobar razmještaj satelita}
		\label{fig:DOPHigh}
	\end{figure}
	
	
	
	Detaljnija podjela pogrešaka tipa 1 nastalih 
	pri određivanju psudo-udaljenosti i na što koja utječe dana je sljedećom tablicom.
	
	\begin{table}[h]\centering
		\caption{Pogreške određivanja pseudo-udaljenosti}
		\begin{tabular}{ |p{3cm}|p{4cm}| }
			\hline
			\rowcolor{lightgray} izvor & utjecaj\\[0.5ex]
			\hline\hline
			\multirow{2}{4em}{satelit} & pogreške orbite  \\ 
			& pogreška sata satelita  \\ 
			\hline
			\multirow{2}{4em}{rasprostiranje signala} & troposferska refrakcija  \\ 
			& ionosferska refrakcija  \\
			\hline
			\multirow{2}{4em}{prijemnik} & pogreške antene \\ 
			& pogreška sata  \\ 
			\hline
		\end{tabular}
	\end{table}
	Utjecaj sistemskih pogrešaka otklanja se modeliranjem ili
	kombinacijom opažanja.
	Korištenjem više prijemnika, otkanjaju se pogreške spacifične za satelite:
	Pogreške specifične za prijemnike otkanja korištenje viška satelita.
	Utjecajem troposfere se najčešće smanjuje modeliranjem,
	a ionosfere korištenjem dva signala različitih frekvencija.
	
	Postoje još i slučajne pogreške nastale zbog aktualnog mjerenja i slučajnog dijela
	višestruke refleksije signala (multipath) nastalog interferencijom 
	direktnog i reflektiranog signala.
	
	U ovom radu, sistemske i slučajne pogreške su otklonjene koristeći RTK-LIB.%RTK LIB u POGLAVLJE 4).
	 Prije konstrukcije ulaza algoritma za određivanja položaja, reducirana je
	pogreška s izvorom u pseudo-udaljenostima.
	Opravdano je pretpostaviti kako ona više nije značajna
	neuzimajući ju u obzir u daljnjem postupku izračuna položaja\label{stranica:greskaOvisisamoOxOpravdano}.
	
	%
	Pogreške tipa 2 mogu imati izvor u dizajnu izvedbe algoritma ili samoj izvedbi:
	\begin{enumerate}
		\item dizajn izvedbe algoritma,
		\item numeričke greške, greške zbog ograničene preciznosti računala,
		 aproksimacije pojedinih vrijednosti,
		\item izvedna algoritma.
	\end{enumerate}
	One se ne modeliraju algoritmima procjene položaja (poglavlje \ref{sec:algoritam}), već prilikom dizajna izvedbe odabranog algoritma (poglavlje \ref{sec:izvedba})
	
	U poglavlju \ref{sec:izvedba}, biti će obrađena analiza pogreške tipa 2 za odabrani algoritam.
	
	\section{Navigation Message}\label{sec:NM} %http://what-when-how.com/gps/gps-details/
	Svaki satelit, uz C/A PRN i P kod, odašilje i dodatne podatke potrebne za ispravo pozicioniranje prijemnika. Odašilje ih u obliku \textit{Navigacijske poruke} koja se šalje zajedno s generiranim C/A PRN kodovima (Slika \ref{Fig:GPSSignal}).
	
	Navigacijska poruka se sastoji od 25 okvira\cite{bookProcessing}.
	Jedan okvir se satoji od 5 podokvira i svaki sadržava vrijeme slanja
	sljedećeg okvira (Slika \cite{GPS:1}). Za slanje cjelokupnog podokvira potrebno je 6 sekundi,
	6 cjelokupnih C/A PRN kodova. Prijemnik je u mogućnosti računati pseudo-udeljenost za novu poziciju satelita svakih
	6 sekundi.
	Za slanje cjelokupne NM, potrebno je 12.5 minuta.
	U nastavku termin poruka koristi se misleći na podprozor.
	
	Prozor sadrži:
	\begin{enumerate}
		\item GPS vremena odašiljanja
		\item signal prijenosa s P na C/A kod (potpoglavlja \ref{Pkod} i \ref{CAkod})
		\item podatke o orbitalnoj putanji satelita
		\item podatke o korekciji sata satelita
		\item almanah statusa svih satelita u sazvježđu
		\item koeficijenti preračunavanja GPS vremena u UTC
		\item ionosferski model korekcije
	\end{enumerate}
	
	\begin{definition}[Universal Time Coordinate (UTC vrijeme)]
		Universal Time Coordinate je vremenski standard zasnovan na međunarodnom atomskom vremenu koji se najčešće koristi u znanstvene i vojne svrhe. Drugi nazivi za taj vremenski standard su ZULU vrijeme i Greenwich Mean Time (GMT).
	\end{definition}
	
	\begin{figure}[H]
		\centering
		\includegraphics[width=0.6\textwidth]{NACONTENT}
		\caption{Pregled strukture prozora navigacijske poruke\cite{GPS:1}}
		\label{Fig:aaa}
	\end{figure}
	Pojedini dijelovi navigacijske poruke pomažu pri otklanjaju pogrešaka tipa 1 spomenutih u 
	potpoglavlju \ref{sec:pogreske}, određivanju
	pseudo-udaljenosti i trenutnoj poziciji satelita.
	Naime, iz podataka o orbitalnoj putanji satelita moguće je za odabrani trenutak izračunati poziciju (koordinate) satelita u orbitalnom koordinatnom sustavu pa i svakom drugom.
	 
	Za razumjevanje ovoga rada, dovoljno je razumjeti sljedeće.
	Prijemnik svakih 6 sekundi ima dovoljno podataka da odredi novu pseudo-udaljenost do istog satelita sve dok on ne prestane biti dostupan. 
	
	\begin{definition}[Dostupnost satelita $\mathnormal{S}$ prijemniku $\mathnormal{T}$]
		\label{stranica:dostupnost}
		Za satelit $\mathnormal{S}$ kažemo da je dostupan prijemniku $\mathnormal{T}$ u trenutku $\mathnormal{t}$ ako je u sljedećih 6 sekundi u mogućnosti izračunati
		pseudo-udaljenost do satelita $\mathnormal{S}$ i konstruirati sljedeću jednadžbu:
		\begin{align}\label{eq:position1}
		d_s = \sqrt{(x-x_s)^{2}+(y-y_s)^{2}+(z-z_s)^{2}}
		\end{align}
		gdje su jedine nepoznanice $(x,y,z)$, tj. koordinate položaja prijemnika.
		$(x_s,y_s,z_s)$ su koordinate položaja satelita. 
	\end{definition}
	
	\section{Proces određivanja položaja}\label{sec:positionProcess}
	U pravilu, prijemnik ima više dostupnih satelita od kojih dobiva poruke u odabranom trenutku. Za određivanje položaja prijemnika u granicama dopuštene točnosti, %TABLICA
	zahtjevaju se barem 4 dostupna satelita\label{stranica:4satelita}.
	
	Kako bi prijemnik odredio svoju poziciju, računa tri nepoznanice: geografsku širinu, duljinu i nadmorsku visinu.
	Neka je $k$ broj vidljivih satelita od prijemnika $\mathnormal{T}$.
	Prijemnik $\mathnormal{T}$ promatrajući poruke dobivene od samo jednog satelita,
	u vremenu $\mathnormal{t}$, izračunava samo jednu pseudo-udaljenost i može konstruirati samo jednu jednadžbu \ref{eq:position1}
	 koja mu omogućava odrediti sferu oko promatranog satelita na kojoj bi se mogao nalaziti (Slika \ref{Fig:1SatelitePosition}).
	
	\begin{figure}[H]
		\centering
		\includegraphics[width=0.4\textwidth]{satellite_distance_13D}
		\caption{Sfera oko promatranog satelita na kojoj bi se prijemnik mogao nalaziti \cite{gps:2}}
		\label{Fig:1SatelitePosition}
	\end{figure}
	Uključujući u izračun pridobivene pseudo-udaljenosti od još jednog satelita dobivamo situaciju prikazanu na Slici \ref{Fig:2SatelitePosition}.
	\begin{figure}[H]
		\centering
		\includegraphics[width=0.6\textwidth]{satellites_distance_23D}
		\caption{Sfere oko 2 promatrana satelita, presjek je kružnica na kojoj bi se prijemnik mogao nalaziti. \cite{gps:2}}
		\label{Fig:2SatelitePosition}
	\end{figure}
	Uključujuči u izračun još jedan satelit, dobivamo situaciju prikazanu na Slici \ref{Fig:3SatelitePosition}.
	
	\begin{figure}[H]
		\centering
		\includegraphics[width=0.6\textwidth]{satellites_distance_33D}
		\caption{Sfere oko 3 promatrana satelita, presjek su 2 točke na kojoj bi se prijemnik mogao nalaziti. \cite{gps:2}}
		\label{Fig:3SatelitePosition}
	\end{figure}
	
	Presjek 3 promatrane sfere su 2 točke na kojoj bi se prijemnik mogao nalaziti.
	Jedna točka se nalazi daleko u svemiru, dok je druga točka točka kandidat pozicije prijemnika. 
	
	Algebarski, rješavamo sljedeći sustav linearnih jednadžbi u $(x,y,z)$ :
	\begin{align}\label{eq:position2}
	 d_1 = \sqrt{(x-x_1)^{2}+(y-y_1)^{2}+(z-z_1)^{2}} \notag \\
	 d_2 = \sqrt{(x-x_2)^{2}+(y-y_2)^{2}+(z-z_2)^{2}} \\
	 d_3 = \sqrt{(x-x_3)^{2}+(y-y_3)^{2}+(z-z_3)^{2}} \notag
	\end{align}
	gdje su $1,2,3$,  3 različita satelita, a $(x_i,y_i,z_i)$ pripadajuće
	koordinate položaja satelita u (ECEF XYZ) koordinatnom sustavu.
	ECEF XYZ koordinatni sustav je prikazan na Slici \ref{Fig:na}. Ishodište (ECEF XYZ) koordinatnog sustava je središte zemlje.
	
	\begin{figure}[H]
		\centering
		\includegraphics[width=0.6\textwidth]{ecefxyz.png}
		\caption{Earth-Centered, Earth-Fixed $\mathnormal{X}$, $\mathnormal{Y}$, $\mathnormal{Z}$ coordinate system ( ECEF XYZ koordinatni sustav ) \cite{GPS:overview}}
		\label{Fig:na}
	\end{figure}
	
	Svaki prijemnik je sposoban izvesti konverziju iz i u koordinata u ECEF XYZ sustavu 
	u i iz geografskih (geografska širina i duljina i nadmorska visina) \cite{GPS:overview}.
	Dakle, prijemniku su potrebna barem 3 dostupna satelita kako bi odredio poziciju.
	Ali ipak na stranici \pageref{stranica:4satelita} se postavlja zahtjev na barem 4. 
	
	Primjetimo kako proces određivanja položaja prijemnika
	indirektno zahtjeva usklađenost satova prijemnika i dostupnih satelita.
	Kako su satovi svih satelita međusobno usklađeni, usklađeni s GPS vremenom (Ukoliko odstupanje postoji odstupanje, biti će zapisano u navigacijskoj poruci koja se može uzeti u obzir prilikom određivanja položaja prijemnika).
	Napomenimo da GPS vrijeme nije jednako UTC vremenu. GPS vrijeme je bilo 0 u 06.01.1980. i određeno je protjecanjem vremena u GPS satelitima, tj. njihovim 
	satovima. 
	
	Satovi prijemnika nisu iste preciznosti kao satovi satelita.
	Prijemnici obično koriste satove preciznosti do otprilike $10^{-6}$ sekundi.
	Pogreška određivanja vremena od $10^{-6}$ sekundi dovodi do pogreške u
	određivanju pseudo-udaljenosti od oko 300 metara.
	Uključijući u izračin i pogrešku sata prijemnika, pseudo-udaljenost modeliramo jednadžbom:
	$$d_i = c\times(t'_i- t_i+ d_T)$$
	gdje $d_T$ predstavlja spomenutu pogrešku.
	Budući da se prilikom određivanja položaja, 
	spomenuta pogreška u oznaci $d_T$ ne mijenja u odnosu na satelit koji se promatra,
	može se izračunati dodavajući ju kao nepoznanicu u sustav jednadžbi \ref{eq:position2}
	Dakle, sustav jednadžbi \ref{eq:position2} prelazi u:
	\begin{align}\label{eq:position3}
	d_1 = \sqrt{(x-x_1)^{2}+(y-y_1)^{2}+(z-z_1)^{2}} +d_T \notag \\
	d_2 = \sqrt{(x-x_2)^{2}+(y-y_2)^{2}+(z-z_2)^{2}} +d_T  \\
	d_3 = \sqrt{(x-x_3)^{2}+(y-y_3)^{2}+(z-z_3)^{2}} +d_T \notag 
	\end{align}
	
	Kako bi za gornji sustav postojala mogućnost pronalaska rješenja,
	uvodi se zahtjev na još barem jedan dostupni satelit, što je ukupno 4 (Stranica \pageref{stranica:4satelita}).
	Dobivamo sljedeći sustav jednadžbi u $(x,y,z,d_T)$:
	\begin{align}\label{eq:1}
	 d_1 = \sqrt{(x-x_1)^{2}+(y-y_1)^{2}+(z-z_1)^{2}} +d_T \notag \\
	 d_2 = \sqrt{(x-x_2)^{2}+(y-y_2)^{2}+(z-z_2)^{2}} +d_T  \\
	 d_3 = \sqrt{(x-x_3)^{2}+(y-y_3)^{2}+(z-z_3)^{2}} +d_T \notag \\
	 d_4 = \sqrt{(x-x_4)^{2}+(y-y_4)^{2}+(z-z_4)^{2}} +d_T \notag
	\end{align}
	
	Upravo opisanom postupkom otklanjamo pogrešku nastalu prilikom 
	određivanja pseudo-udaljenosti 
	s izvorom u pogrešci sata prijemnika.
	U praksi se može koristiti još veći broj dostupnih satelita što poboljšava 
	preciznost pozicioniranja prijemnika.

\chapter[Algoritam procjene položaja (APP)]{Algoritam procjene položaja u domeni navigacijske primjene}\label{sec:algoritam}

\textit{Algoritam procjene položaja u domeni navigacijske primjene} (APP)
smatramo svakim algoritmom koji za sustav jednadžbi \eqref{eq:1}
određuje nepoznatu poziciju prijemnika u koordinatama $(x,y,z)$.
Broj jednadži sustava može biti i veći od 4. Tada govorimo o prezasićenim sustavima.
Ovisno o odabiru, APP se može temeljiti na rješavanju sustava nelinearnih jednadžbi pronalaženjem rješenja pomoću (1) metode najmanjih kvadrata (Newton-ova metoda),
(2) pomoću zatvorene formule, (3) metode najbližeg susjeda ili (4) vjerojatnosne metode \cite{math:positioning}. 

Općenito, rješava se moduficiran sustav jednadžbi \ref{eq:1}:\\
\begin{align}\label{eq:new}
d_1 = \sqrt{(x-x_1)^{2}+(y-y_1)^{2}+(z-z_1)^{2}} +d_T + v_1\notag \\
d_2 = \sqrt{(x-x_2)^{2}+(y-y_2)^{2}+(z-z_2)^{2}} +d_T + v_2 \\
d_3 = \sqrt{(x-x_3)^{2}+(y-y_3)^{2}+(z-z_3)^{2}} +d_T + v_3\notag \\
d_4 = \sqrt{(x-x_4)^{2}+(y-y_4)^{2}+(z-z_4)^{2}} +d_T + v_4\notag
\end{align}
u koji uključejemo nepoznati parametar $(v_1,v_2,v_3,v_4)$, dodatnu pogreška
izračuna.

Uz oznake 
\begin{align}
\mathbf{\rho} := (d_1, d_2, d_3, d_4)^T \\ 
\mathbf{x} := (x,y,z,d_T)^T \\ 
\mathbf{s}_i := (x_i,y_i,z_i)^T \\ 
\mathbf{h} (\mathbf{x}) := 
\begin{bmatrix}
||(s_1-\mathbf{x}_{1:3})|| + x_4\\
||(s_2-\mathbf{x}_{1:3})|| + x_4\\
||(s_3-\mathbf{x}_{1:3})|| + x_4\\
||(s_4-\mathbf{x}_{1:3})|| + x_4
\end{bmatrix} \\
\mathbf{v} := (v_i,v_2,v_3,v_4)^T \label{eq:v}
\end{align}%
prelazi u
\begin{align}\label{eq:matrix}
\mathbf{\rho} = \mathbf{h}(\mathbf{x})+\mathbf{v}
\end{align}%
\section{Iterativna metoda najmanjih kvadrata}

Primjetimo kako je $\mathbf{h}(\mathbf{x})$ jednako pravom vektoru udaljenosti između
satelita i prijemnika za prave vrijednosti $\mathbf{x}$,$\bar{\mathbf{x}}$.\\
Na ovoj razini, algoritam oderđivanja pozicije se ne bavi pogreškom tipa 2,
već samo pogreškom tipa 1 (stranica \ref{stranica:greskaOvisisamoOxOpravdano}).
Također, može se pretpostavti kako su otklonjene sve pogreške tipa 1 koje imaju izvor 
u izračunu pseudo-udaljenosti (stranica \ref{stranica:greskaOvisisamoOxOpravdano}).
Ostaje samo modelirati pogreške koje imaju za izvor trenutni položaj satelita dostupnih za
izračunavanje željenog položaja $\bar{\mathbf{x}}_{(1:3)}$.
U tu svrhu modeliramo vektor pogrešaka $\mathbf{v}$, funkcijom $\mathbf{p}(\mathbf{x})$ koja ovisi o nepoznatom parametru $\mathbf{x}$.
Uz oznaku $\mathbf{y} := \rho$ 
 jednadžba  \ref{eq:matrix} prelazi u
\begin{align}\label{eq:matrix2}
\mathbf{y} = \mathbf{\rho} = \mathbf{h}(\mathbf{x})+ \mathbf{p}(\mathbf{x})
\end{align}%
Preciznije,
član $\mathbf{p}(\mathbf{x})$ modelira pogrešku razlike u procjeni parametra $\mathbf{x}$ od stvarne vrijednosti.
Što je bolja aproksimacija potrebnih vrijednosti za izračun rješenja matrične jednadžbe
\ref{eq:matrix} točnija, to je $\mathbf{p}(\mathbf{x})$ 
bliže nuli za pravu vrijednost pozicije prijemnika $\bar{\mathbf{x}}$.
Tada aproksimaciju za $\bar{\mathbf{x}}$, u oznaci $\hat{\mathbf{x}}$, pronalazimo tražeći nultočke funkcije $\mathbf{p}(\mathbf{x})$.
U praksi je uobičajeno da mjerenja sadrže pogreške. Tada $\mathbf{p}(\mathbf{x})$ uopće ne mora imati 
nultočke i $\hat{\mathbf{x}}$ ne možemo pronaći tražeći nultočke funkcije $\mathbf{p}(\mathbf{x})$.

Ideja metode najmanjih kvadrata je pronalazak $\hat{\mathbf{x}}$ tražeći minimum $\mathbf{p}(\mathbf{x})$, tj.
\begin{align}\label{eq:minimization}
	\hat{\mathbf{x}} = \text{arg min}_\mathbf{x} \mathbf{p}(\mathbf{x})^T\mathbf{p}(\mathbf{x})
\end{align}
Problem opisan jednadžbom \eqref{eq:minimization} nije linearan pa
ne postoji općeniti način pronalaska njegovog rješenja.

U slučaju da su funkcija koju treba minimiziati i početna vrijednost $\mathbf{x}_0$
(iterativnog postupka) dovoljno dobre (vidi: dodatak \ref{appendix:aTay}), rješenja problema \ref{eq:minimization} možemo
dobiti iterativnim postupakom.
Ideja iterativnog postupka je počevši s $\mathbf{x}_0$ računati $\mathbf{x}_1, \mathbf{x}_2, \hdots $ sve dok se novoizračunate vrijednosti ne prestanu mijenjati ili postanu dovoljno bliske prethodnoj, tj.
$\left \| x_{k} - \mathbf{x}_{k-1}\right\| < \delta$ za dovoljno male $\delta > 0$.
$\delta$ još nazivamo i zaustavni kriterij.

Jedan iterativni postupak rješavanja problema \ref{eq:minimization} je
pomoću Newton-Gaussove metode (iterativna metoda najmanjih kvadrata).
Newton-Gaussova metoda linearizira $\mathbf{p}(\mathbf{x})$ u okolini od $\mathbf{x_k}$ razvojem u  $\mathbf{p}(\mathbf{x})$ u Taylorov red\label{stranica:NGLin}:
\begin{align}\label{eq:approx}
	\mathbf{p}(\mathbf{x_k}+ \Delta \mathbf{x_k}) \approx \mathbf{p}(\mathbf{x_k}) + \mathbf{p}'(\mathbf{x_k})\cdot \Delta \mathbf{x_k}
\end{align}
$\Delta \mathbf{x_k}$ se odabire na način tako da
$$lim_{k \to \infty} \left( \mathbf{p}(\mathbf{x_k}) \right) = 0$$ za $\mathbf{p}(\mathbf{x_{k+1}}) := \mathbf{p}(\mathbf{x_k}+ \Delta \mathbf{x_k}) \approx \mathbf{p}(\mathbf{x_k}) + \mathbf{p}'(\mathbf{x_k})\cdot \Delta \mathbf{x_k}$.
Dakle, $ \Delta \mathbf{x_k} $ odabiremo tako da tražimo rješenje jednadžbe
\begin{align}\label{eq:minDelta}
	\mathbf{p}(\mathbf{x_k}) + \mathbf{p}'(\mathbf{x_k})\cdot \Delta \mathbf{x_k} = 0
\end{align}
Označimo sada s $J_k := \mathbf{p}'(\mathbf{x_k}) = \mathbf{h}'(\mathbf{x_k})$.
Sada \ref{eq:minDelta} možemo zapisati u
obliku 
\begin{align}\label{eq:minDelta2}
	J_k \Delta \mathbf{x_k} = -\mathbf{p}(\mathbf{x_k})
\end{align}
koji predstavlja prezasićen sustav linearnih jednadžbi čije je
rješenje definirano s  
\begin{align}\label{eq:minDeltaRj}
\Delta \mathbf{x_k} = - (J_k^TJ_k)^{-1}J_k^T \mathbf{p}(\mathbf{x_k})
\end{align}
Sada, $\mathbf{x_{k+1}}$ dobivamo pomoću jednadžbe 
\begin{align}\label{eq:iter}
	\mathbf{x_{k+1}} = \mathbf{x_{k}} - (J_k^TJ_k)^{-1}J_k^T \mathbf{p}(\mathbf{x_k})
\end{align}
Prilikom izvedbe algoritma, potrebano je dobro odrediti početnu vijednost $\mathbf{x_0}$, te kasnije iterirati po formuli \ref{eq:iter}.
Ukoliko odaberemo dovoljno dobar $\mathbf{x}_0$, dovoljno blizu rješenju i
ako je druga derivacija od $p$ u točki $\bar{\mathbf{x}}$ dovoljno mala,
niz $x_0,x_2, \hdots$ konvergira prema $\bar{\mathbf{x}}$.

Algoritam iterativne metode najmanjih kvadrata definiran je pseudo-kodom \ref{code:iterLSM}.
\begin{algorithm}[H]
	\KwData{ $\mathbf{p(\mathbf{x})}, \mathbf{x}_0, \delta$ }
	\KwResult{ $\hat{\mathbf{x}} $ }
	$k = 0$ \;
	\While{$ \left \| \mathbf{x}_{k} - \mathbf{x}_{k-1}\right\| \geq \delta $}{
		$J_k = \mathbf{p}'(\mathbf{x}_k)$ \;
		$\Delta \mathbf{x}_k = -\frac{J_k}{\mathbf{p}(\mathbf{x}_k)}$ \;
		$\mathbf{x}_{k+1} =\mathbf{x}_k + \Delta \mathbf{x}_k$ \;
	$k ++$\;
	} 
	$\hat{\mathbf{x}} = \mathbf{x_k}$
\caption{Iterativna metoda najmanjih kvadrata}
\label{code:iterLSM}
\end{algorithm}

Ukoliko gornji algoritam primjenjujemo za oderđivanje pozicije entiteta i nemamo boljih kandidata za $\mathbf{x}_0$ mogu uzeti koordinate središta zemlje.
Naime, jednadžbe za određivanje položaja su dovoljno blizu linearnim.

Ukoliko znamo da su vrijednosti koje koristimo za konstrukciju
jednadžbi za određivanje položaja \eqref{eq:1} za neke jednadžbe točnije nego za druge,
pametno je dati veći značaj tim jednadžbama nego ostalima.
Svakoj jednadžbi se pridaje težina $\sigma_i$ koja je proporcionalna točnosti 
vrijednosti koje se koriste u njezinoj konstrukciji.
Najčešće se težine modeliraju preko kovarijancone matrice
vektora pogrešaka $\mathbf{v}$ \eqref{eq:v},u oznaci $\Sigma : = cov(\mathbf{v})$, pa minimizacijski problem \ref{eq:minimization} prelazi u %
\begin{align}\label{eq:minimisation2}
\hat{\mathbf{x}} = \text{arg min}_\mathbf{x} \mathbf{p}(\mathbf{x})^T \Sigma \mathbf{p}(\mathbf{x})
\end{align}%
Sada, algoritam \ref{code:iterLSM} prelazi u algoritam \ref{code:iterLSMW}.

\begin{algorithm}[h]
	\KwData{ $\mathbf{p(\mathbf{x})}, \mathbf{x}_0, \delta$, $\Sigma$ }
	\KwResult{ $\hat{\mathbf{x}} $ }
	$k = 0$ \;
	\While{$ \left \| \mathbf{x}_{k} - \mathbf{x}_{k-1}\right\| \geq \delta $}{
		$J_k = \mathbf{p}'(\mathbf{x}_k)$ \;
		$\Delta \mathbf{x}_k = -\frac{\Sigma^\frac{1}{2}J_k}{\Sigma^\frac{1}{2}(\mathbf{p}(\mathbf{x}_k))}$ \;
		$\mathbf{x}_{k+1} =\mathbf{x}_k + \Delta \mathbf{x}_k$ \;
		$k ++$\;
	} 
	$\hat{\mathbf{x}} = \mathbf{x_k}$
	\caption{Iterativna metoda težinskih najmanjih kvadrata}
	\label{code:iterLSMW}
\end{algorithm}
Procjenitelj za $\bar{\mathbf{x}}$ dobiven težinskom metodom najmanjih kvadrata \ref{eq:minimisation2} ima najmanju varijancu među svim procjeniteljima za
$\bar{\mathbf{x}}$. Ukoliko je vektor pogrešaka $\mathbf{v}$ normalno
distribuiran, procjenitelj \ref{eq:minimisation2} postaje procjenitelj
metode najbližeg susjeda \eqref{sec:MLE} (MLE procjenitelj).
Izračun od $J_k$ se može naći u prilogu \ref{appendix:aTay}.

Težinska metoda najmanjih kvadrata daje najboljeg procjenitelja za $\bar{\mathbf{x}}$
uz poznatu distribuciju vektora pogrešaka, tj. kovarijancone matrice $\Sigma$.
Prilikom korištenja težinske metode najmanjih kvadrate, potrebno je pripazati na 
velike pogreške u određivanju vrijednosti pomoći kojih se gradi sustav jednadžbi 
\ref{eq:1} i netipične vrijednosti ("outlinere") koji se uklanjaju prije primjene algoritma.

Sljedeće poglavlje donosi izvedbu upravo opisanog algoritma \ref{code:iterLSMW} i 
analizu njegove točnosti. Prije toga navedimo jednu zanimljivu posljedicu dobivenu
analizom pogreške metode najmanjih kvadrata i pregled ostalih metoda za rješavanje
sustava \ref{eq:1}.

\subsection{Analize pogreške metode najmanjih kvadrata}
Uz oznake kao do sada, neka $\bar{\mathbf{y}}$ predstavlja prave udaljenosti između satelita i promatranog entiteta i $\hat{\mathbf{y}}$ izračunate psudo-udaljenosti. 
Vrijedi
$\hat{\mathbf{y}} = \bar{\mathbf{y}} + \Delta \mathbf{y}$.
Promatramo idealan slučaj. Neka je
metoda najmanjih kvadarata konvergirala k 
$\hat{\mathbf{x}} = \bar{\mathbf{x}} + \Delta \mathbf{x}$, tj. 
$\mathbf{x}_k' = \hat{\mathbf{x}}$ je rješenje dobiveno metodom najmanjih kvadrata uz $\delta = 0$ i $\forall m \geq k', \mathbf{x}_m = \mathbf{x}_{m+1}$.
Uvrštavanjem $\mathbf{x}_k := \hat{\mathbf{x}}$ i $\hat{\mathbf{y}}$ u jednadžbu \ref{eq:iter} dobivamo
\begin{align*}
	\mathbf{x_{k+1}} &= \mathbf{x_{k}} - (J_k^TJ_k)^{-1}J_k^T \mathbf{p}(\mathbf{x_k}) \\
	\mathbf{x_{k+1}} - \mathbf{x_{k}} &= - (J_k^TJ_k)^{-1}J_k^T \mathbf{p}(\mathbf{x_k}) \\
	0 &= - (J_k^TJ_k)^{-1}J_k^T \mathbf{p}(\bar{\mathbf{x}} + \Delta \mathbf{x}) \\
	0 &= (J_k^TJ_k)^{-1}J_k^T (\mathbf{h}(\bar{\mathbf{x}} + \Delta \mathbf{x}) -(\bar{\mathbf{y}} + \Delta \mathbf{y}))\\
\end{align*}
Matica $J_k$ predstavlja funkciju koja ovisi o parametru $\mathbf{x}$ i nije konstantna.
Kako 
se pretpostavlja da je $\Delta \mathbf{x}$ mali, $J := J_k$ se može smatrati gotovo konstantnom u susjedstvu od $\bar{\mathbf{x}}$ radijusa $\Delta \mathbf{x}$.
Stoga se $\mathbf{h}$ u okolini točke $\bar{\mathbf{x}}$ može linearizirati na sljedeći način:
$\mathbf{h}(\mathbf{x}+\delta) = \mathbf{h}(\mathbf{x}) + J \delta, \delta > 0$
Dobivamo
\begin{align*}
0 &= (J^TJ)^{-1}J^T (\mathbf{h}(\bar{\mathbf{x}}) + J\Delta \mathbf{x} -(\bar{\mathbf{y}} + \Delta \mathbf{y}))\\
0 &= (J^TJ)^{-1}J^T (J\Delta \mathbf{x} - \Delta \mathbf{y})\\
(J^TJ)^{-1}J^T J\Delta \mathbf{x}& = (J^TJ)^{-1}J^T \Delta \mathbf{y}\\
\Delta \mathbf{x} &= (J^TJ)^{-1}J^T \Delta \mathbf{y}\\
\end{align*}
Ukoliko pretpostavimo normalnost pogreške izračunavanja pseudo-udaljenosti,
\newline $\Delta \mathbf{y} \sim N(0,\Sigma)$ slijedi
\begin{align}\label{eq:xerrorDistr}
	\Delta \mathbf{x} \sim N(0,(J^TJ)^{-1}J^T\Sigma J(J^TJ)^{-1})
\end{align}
Uz $\Sigma = \sigma^2I$, $\Delta \mathbf{x} \sim N(0,\sigma^2(J^TJ)^{-1})$.
U kontekstu satelitske navigacije, $(J^TJ)^{-1}$ se naziva DOP matrica (engl. Dilution of Precision).
Iz DOP matrice moguće je izvesti različite mjere kvalitete "zviježđda" satelita u danom trenutuku za danu poziciju.
\begin{enumerate}
	\item GDOP = $\sqrt{tr(J^TJ)^{-1}}$
	\item PDOP = $\sqrt{tr((J^TJ)^{-1}_{(1:3,1:3)})}$
	\item HDOP = $\sqrt{tr((J^TJ)^{-1}_{(1:2,1:2)})}$
	\item VDOP = $\sqrt{(J^TJ)^{-1}_{(3,3)}}$
	\item TDOP = $\sqrt{(J^TJ)^{-1}_{(4,4)}}$
\end{enumerate}
Opširnije o mjerama kvalitete "zviježđda" moguće je naći u dodatku \ref{appendix:DOP}.

Uz dane pretpostavke Jakobijeva matrica funkcije $\mathbf{h}$, $J$ može reći mnogo o kvaliteti 
određivanja položaja za sustav jednadžbi \ref{eq:1}, veličini pogreške određivanja.
Izračuni gornjih mjera su onoliko točni koliko su pretpostavke
o jednakosti varijanci za $\Delta \mathbf{y}$ i $\Delta \mathbf{x}$
istinite.
%DODATI KAKO PRAVO DEFINIRATI h DA SE dobije J(4,4) TDOP u appendix.
\section{Metoda pomoću zatvorene formule}
%MIAMIAMIAMIAMIAMIAMAIMAIMAI

\section{Metoda najbližeg susjeda}\label{sec:MLE}


\section{Vjerojatnosne metode}






\chapter{Dizajn i izvedba algoritma, procjena točnosti}\label{sec:izvedba}
%DODATI U LITERATURU str 366.
%https://web.math.pmf.unizg.hr/~singer/num_alg/num_anal.pdf
Kratki opis koraka koje smo napravili prilikom izvedbe.
\subsection{Zahtjevi algoritma}
Ulaz (U rinex) + uzorak za usporedbu(RINEX)
\subsubsection{RINEX}
-objašnjenje pojma + primjeri. + otkuda nam.
\subsubsection{Programski određen GPS prijemnik}
+zašto i kako smo ga koristili. i dobili podatke. (User manual?)

\subsection{Izvedba}
R kod
\subsection{Procjena točnosti}
+uočene pogreške, uvjetovanost matrice.
\section{Poboljšani algoritam i njegova izvedba}
\subsection{Numerička metoda koja dovodi do poboljšanja}
Zašto , kako smo došli do toga!
\subsection{Izvedba}
R kod
\subsection{Procjena točnosti}
kako već

\section{Usporedba osnovnog i poboljšanog algoritma} + zaključak zašto je bolji.

\section{Zaključak}


%\label{stranica}
%Na stranici \pageref{stranica} se nalaza slika u \textbf{png} formatu.

\bibliographystyle{babamspl} % babamspl ili babplain

% U datoteku diplomski.bib se stavljaju bibliografske reference
% Bibliografske reference u bib formatu se mogu dobiti iz MathSciNet baze, Google Scholara, ArXiva, ...
\bibliography{diplomski}

\pagestyle{empty} % ne zelimo brojanje sljedecih stranica

% I na koncu idu sazeci na hrvatskom i engleskom

\begin{sazetak}
Satelitsko određivanje položja predstavlja temeljnu
tehnologiju rastućeg broja tehnoloških i društveno-ekonomskih sustava.
Kvaliteta njihovih
usluga određena je točnoću procjene položja
satelitskim sustavima.
Programski određen radioprijamnik za satelitsku navigaciju
procesira signale za određivanje položja i podatke
iz navigacijske poruke
u tri osnovne domene: radiofrekvencijskoj, u domeni osnovnog frekvencijskog
područja te u domeni navigacijske primjene.
Ovaj rad analizira postupak procjene položaja
u domeni navigacijske primjene. U tu svrhu, koriste se na osobnom računalu
izveden programski određen GPS prijamnik i ulazni podatci
o opaženim pseudoudaljenostima spremljeni
u RINEX podatkovnom formatu.
Analiza korištenog algoritma procjene položaja
temelji se na izmjerenim pseudoudaljenosti (Sanz Subirana et al, 2013, Chapter 6.1)
te se otkrivaju potencijalne slabosti algoritma
s učincima na točnost procjene položja. Na kraju, predlažu se poboljšanja
algoritma te ih se izvodi u programskom okruženju R. 
Poboljšanja algoritma su vrednovana komparativnom analizom obilježja
poboljšanog i izvornog algoritma.
\end{sazetak}

\begin{summary}
In this ...
\end{summary}

% te zivotopis

\begin{cv}
Dana ...
\end{cv}

\appendix

\chapter{Taylorov red potencija}\label{appendix:aTay}
Primjetimo kako smo na stranici \pageref{stranica:NGLin} pretpostavili
kako će za rezidualnu funkciju $\mathbf{p}(\mathbf{x})$ postojati njezin 
razvoj u Taylorov red oko svake točke $\mathbf{x_k}$. Ipak,
Taylorov red nije definiran za svaku funkciju na $\R^n, n \in \N$.
Prilikom primjene Iterativne metode najmanjih kvadrata, treba zahtjevati da 
funkcija $\mathbf{p}(\mathbf{x})$ i točka $x_k$ zadovoljava uvjete definicije razvoja funkcije u Taylorov red oko točke $x_k$ \cite{math:tay}.
\begin{definition}
	Naka je $f: \mathbf{I} \to \R $ funkcija klase $C^\infty(\mathbf{I})$ definirana
	na otvorenom intervalu $\mathbf{I} \subseteq \R^n$ i neka je $c \in \mathbf{I}$.
	Red potencija
	\begin{align}
	T \left[f,c\right] := \sum_{n=0}^{\infty} \frac{f^{(n)}}{n!} \left(x - c\right)^n
	\end{align}
	nazivamo \textbf{Taylorov red} funkcije $f$ oko točke $c$.
\end{definition}%

Također,
pretpostavlja se kako je 
\begin{align}\label{eg:a1}
	\mathbf{p}(\mathbf{x_{k+1}}+\Delta \mathbf{x}_k) = T \left[\mathbf{p},x_k \right]
\end{align}
što općenito nije točno.
Naime, Taylorov red $T \left[f,c\right] $ funkcije 
$f \in C^\infty(\mathbf{I})$ nužno ne konvergira za svaki $x \not = c, x \in \mathbf{I}$
ili može konvergirati prema nekoj drugoj funkciji.
Uvjete pod kojima zaista vrijedi \ref{eg:a1} opisani su teoremima u nastavku.

\begin{definition}[Analitička funkcija]
	Za $f \in C^\infty(\mathbf{I})$ kažemo da je \textbf{analitička u točki} $c \in \mathbf{I}$ ako njezin Taylorov red:
	\begin{align*}
		T \left[f,c\right] := \sum_{n=0}^{\infty} \frac{f^{(n)}}{n!} \left(x - c\right)^n
	\end{align*}
	
	ima radijus konvergencije $R > 0$ i ako postoji $0 < \delta \leq R$ takav da vrijedi 
	\begin{align*}
	f(x) = T \left[f,c\right] := \sum_{n=0}^{\infty} \frac{f^{(n)}}{n!} \left(x - c\right)^n,
	\forall x \in \left < c-\delta, c+\delta \right > \cap \mathbf{I}
	\end{align*}
	U oznaci: $f \in C^\omega(\mathbf{I})$.
\end{definition}
\begin{thm}\label{thm:konv1}
	Neka je $\sum_{n=0}^{\infty} a_n \left(x - c\right)^n$ red potencija s radijusom konvergencije $R > 0$. Za $\mathbf{I} := \left < c-R, c+R\right >$,
	funkcija $f: \mathbf{I} \to \R$ definirana s 
	\begin{align}
		f(x) := \sum_{n=0}^{\infty} a_n \left(x - c\right)^n
	\end{align}
	je analitička na čitavom $\mathbf{I}$. Nadalje, za svaki $\alpha \in \mathbf{I}$ 
	pripadni Taylorov red
	\begin{align}
		T \left[f,\alpha \right] = \sum_{n=0}^{\infty} \frac{f^{(n)}}{n!} \left(x - \alpha \right)^n
	\end{align}
	ima radijus konvergencije $\rho \leq R - (c - \alpha)$ i vrijedi 
	\begin{align}
	f(x) = T \left[f,\alpha \right] = \sum_{n=0}^{\infty} \frac{f^{(n)}}{n!} \left(x - \alpha \right)^n
	\end{align}
\end{thm}


\begin{thm}\label{thm:konv}
	Naka je $f: \mathbf{I} \to \R $ funkcija klase $C^\infty(\mathbf{I})$ definirana
	na otvorenom intervalu $\mathbf{I} \subseteq \R^n$.
	Tada je $f \in C^\omega(\mathbf{I}$ ako i samo ako za svaki $c \in \mathbf{I}$ postoje
	$\delta > 0$ i konstante $C > 0$ i $ r > 0 $ takve da za sve $n \in \Z_+$ vrijedi:
	\begin{align}
		\left | f^{n}(x)  \leq  C \frac{n!}{r^n}  \right | 
		\forall x \in \mathbf{J} := \left < c-\delta, c+\delta \right > \cap \mathbf{I}
	\end{align}
U tom slučaju $f(x) = T \left[f,c\right](x) $ $
\forall x \in \left < c-r, c+r \right > \cap \mathbf{J}$.
\end{thm}
Za primjenu iterativne metode
najmanjih kvadrata $\mathbf{p}$ mora biti klase $C^\infty(\mathbf{I})$ gdje je $\mathbf{I}$ unija otvorenih okolina oko svih izračinatih $\mathbf{x}_k, k \in \N$, osim zadnjega.
Također, otvorena okolina oko $\mathbf{x}_k$  mora barem sadržavaki otvorenu kuglu $K(\mathbf{x}_k,\Delta \mathbf{x}_k)$ i za $\mathbf{p}$ mora vrijediti teorem \ref{thm:konv1} ili  \ref{thm:konv}.

\chapter{Jakobijeva matrica funkcije $\mathbf{h}$, $J$}
Iz jednakosti \ref{eq:matrix2} i $J = \mathbf{p}(\mathbf{x})$ dobivamo
\begin{align}
J = \frac{\partial \mathbf{h}}{\partial \mathbf{x}}
\end{align}
Za 
\begin{align}
\mathbf{h} (\mathbf{x}) := 
\begin{bmatrix}
||(s_1-\mathbf{x}_{1:3})|| \\
||(s_2-\mathbf{x}_{1:3})|| \\
||(s_3-\mathbf{x}_{1:3})||\\
\end{bmatrix} 
\end{align}
dobivamo
\begin{align}
J = \begin{bmatrix}
\frac{\partial}{\partial \mathbf{x}} ||(s_1-\mathbf{x}_{1:3})|| \\
\frac{\partial}{\partial \mathbf{x}} ||(s_2-\mathbf{x}_{1:3})||\\
\frac{\partial}{\partial \mathbf{x}} ||(s_3-\mathbf{x}_{1:3})|| 
\end{bmatrix}%
= - \begin{bmatrix}
\frac{(s_1-\mathbf{x}_{1:4})^T}{||(s_1-\mathbf{x}_{1:3})||} \\
\frac{(s_2-\mathbf{x}_{1:4})^T}{||(s_1-\mathbf{x}_{1:3})||}\\
\frac{(s_3-\mathbf{x}_{1:4})^T}{||(s_1-\mathbf{x}_{1:3})||} 
\end{bmatrix} = (J_n(1:3,1:3))
\end{align}
za $\hat{\mathbf{x}} = \mathbf{x}_n$
\chapter{Mjere kvalitete "zviježđda"}\label{appendix:DOP}

\end{document}